\chapter{Integración} 

En este apartado estaremos abordando la discusión en dos ejes principales:

\begin{itemize}
\item Reflexiones en torno al software que es arte y al software que produce arte y cómo esto tiene consecuencias en el proceso artístico y de investigación que se ha realizado hasta el momento. 
\item La posible resolución de la brecha que existe entre la escritura que genera programas de computadora, investigación y acción creativa. 
\end{itemize}

En este sentido, este apartado aborda la relación entre la escritura de/con software y la acción que se activa cuando se ejecuta esa escritura. 

\section{La escritura como rodeo}

Hasta el momento hemos abordado tangencialmente la temática de la escritura. Este apartado describe de manera recursiva el proceso por el cuál fue escrita la versión expandida de esta investigación. Si bien hay una versión en formato PDF/impresión, también hay una versión que intercala de una manera integrada los elementos que hemos mencionado hasta el momento: audio, imagen y texto. Para realizar esta labor, retomaremos la noción de síntesis granular para desplazar el ordenamiento de estos elementos a una entidad históricamente referida en la escritura de software para hacer audio y tecnológicamente integradora para esta investigación: el generador unitario. En este sentido no persigo el diseño de un documento con el mismo texto de la investigación embebido y acompañado con audio o imágenes de fondo; busco explorar la posibilidad de escribir módulos para el reordenamiento del material en el espacio y tiempo integrados desde el elemento tecnológico más pequeño que las puede producir. La suma de estos elementos puede generar una densidad que transmita en la dimensión de lo sensible, las reflexiones vertidas en esta investigación. 


Recursividad
Documentos que se ejecutan 

\section{Segunda sección}

%% realización

La realización de este proyecto de investigación y la integración de los elementos que lo componen puede complementarse con una navegación estilo videojuego. La noción de cámara y algunos elementos gestuales que la componen como posición y rotación permite una navegación que toma en cuenta las decisiones espaciales del usuario. Aunado a esto, es posible extender las posibilidades del espacio tridimensional con distintos tipos de controles que de acuerdo a la necesidad, centran su comportamiento dependiendo del objetivo: vuelo, orbitación o exploración como videojuegos en primera persona\footnote{En este caso refiero a la convención de los videojuegos de disparos en primera persona (FPS). Para la enunciación de esta perspectiva he decidido prescindir de la palabra disparos por su carácter bélico. La exploración bajo esta perspectiva no necesariamente se adscribe a la búsqueda o eliminación de algún objetivo.}

Juegos que busca la mejor panorámica, por ejemplo pokemon y fotografías. 

\section{Tercera sección} 

% Quitar esto,  pensar más en un libro de autor
% ¿Cómo convertir tu tesis en un libro?

%\subsection{Resultados}

% Pienso que esto podría sustituirse por integración pero no sé, siento que aquí hay algo redundante

%Como un repaso de lo que está escrito antes, apuntes para las conclusiones

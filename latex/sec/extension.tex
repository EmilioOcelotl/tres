\chapter{Extensión y escritura}

En este capítulo hablaremos de dos aspectos que detonan reflexión referente a la escritura de módulos de software personalizados.

\begin{itemize}
\item La escritura con lenguajes de programación como una posibilidad para la reconfiguración de las distinciones entre usuarios y desarrolladores.
\item El papel de la diversidad tecnológica en la extensión de los marcos de trabajo y la resistencia a la escalada de recursos computacionales
\end{itemize}

En este sentido, destaco el papel que tiene la extensión de las funcionalidades de marcos de trabajo y lenguajes de programación. En particular, es de interés de este trabajo destacar la ampliación de estas posibilidades de cara a las consecuencias poéticas de piezas construídas con lenguajes de programación. Estas librerías adicionales también pueden perseguir, de manera paralela, objetivos funcionales que pueden medirse de alguna manera. 

\section{Modulos personalizados}

Escritura de bibliotecas para alcanzar una expresividad intermedia.
¿Qué implica escribir módulos personalizados?
Primero que no existan antecedentes y que la idea en lo general pueda al menos realizarse técnicamente.

\section{Reconfiguración de distinciones}

Acción: reconfigurar la distinción usuario/desarrollador-escritor
PLab como casos. Espacio de encuentro y tecnodiversidad
Declaración/acción

\section{Acción e idea}

El código como intepretación y como partitura
Valoraciones del código: elegancia y ofuscación
Encuentros y brechas entre escritura de código, de productos académicos y de algo más abierto.

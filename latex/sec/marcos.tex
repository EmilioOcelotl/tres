
\chapter{Marcos de trabajo}

% Este apartado se parece a una sección de los antecedentes. Creo que podría mover algo de eso por acá para hacer el repaso histórico de las herramientas 
% En este apartado se puede abordar la contradicción tecnológica 

% Antes este apartado se llamaba música algorítmica, supongo que quise retomar las discusiones sobre el uso amplio de estas plataformas es decir, fuera de supercollider 

Legado que hila el pensamiento musical algorítmico. Plataformas como Music N y el rastreo que hace Ge Wang sobre este tema.

\section{Legado Audio}

\subsection{Marcos de audio}

A continuación, haremos un breve rastreo de ideas centrales en la escritura de software orientado a la creación musical. El objetivo de este apartado consiste en describir y detectar la presencia del concepto unidad generadora en diversos programas orientados a la generación musical, de entre los cuales está una de las librerías que la presente investigación implementa.

El punto de partida de esta indagación es MUSIC N, proyecto de Max Mathews que sería el parteaguas del paradigma de la música por computadora. Uno de los primeros casos de esta instancia es el principio de programas como Max/MSP y PureData, proyectos representativos de la programación gráfica presente en flujos de trabajo gráfico actuales como TouchDesigner.

Como una observación adicional, es importante detectar instancias de la programación gráfica y de las ideas principales de la música por computadora en plataformas con giros programáticos particulares. Tal es el caso de OpenMusic, que de manera específica está basado en Common Lisp.

Desde la perspectiva de la programación escrita, destacamos el papel que ha tenido SuperCollider en la extensión del paradigma de la música por computadora en la actualidad. Señalamos la importancia de SuperCollider como el motor bajo el cual se pueden ejecutar entornos de programación al vuelo como Tidal Cycles o FoxDot. 

Estuary es un caso adicional que permite establecer un puente entre Tidal Cycles como un entorno que ejecuta SuperCollider como motor de audio y el navegador como tecnología multiplataforma sin instalaciones. Esta plataforma utiliza secuenciadores basados en la sintaxis de Tidal Cycles pero a diferencia del entorno que se puede instalar de manera local en una computadora, Estuary utiliza al navegador como motor de audio. 

Trazamos estas relaciones para establecer una relación continua y presente entre los entornos anteriormente descritos y una de las librerías utilizadas en los casos de estudio de esta investigación: Tone.js. En este sentido, consideramos importante la adscripción a los principios de la música por computadora para encontrar soluciones personalizadas para el navegador.


\subsection{Jacktrip y la música conectada}

Una parte de los antecdentes de este proyecto se vinculan con la actividad del colectivo RGGTRN\footnote{\url{https://rggtrn.github.io/}. Consultado el \today} y del LiveCodeNet Ensamble\footnote{\url{https://livecodenetensamble.wordpress.com/}. Consultado el \today}

RGGTRN es un colectivo de música por computadora fundado por Luis N. Del Angel y Emilio Ocelotl. Posteriormente se unieron Marianne Teixido y Jessica A. Rodríguez. Como parte de un ejercicio lúdico y de reflexión, el colectivo explora la improvisación audiovisual realizada por medio de código de programación, con una relación al contexto Latinx de sus participantes.

% Tesis de Luis
% Bellacode
% Saborítmico
% Tesis de maestría de hernani
% Artículo de Hernani en algún lado
% Artículos de Jessica ? 

raspis conectadas y jacktrip > el trabajo realizado por CCRMA y en general 

La labor del colectivo Radiador

% La radio y la transmisión de sonido con Icecast 

Sonobus y la resolución de problemas de streaming en tiempos de pandemia

\begin{itemize}
\item SuperCollider
\item CommonMusic 
\item Incluso plataformas de más alto nivel de lo que se trata en esta investigación como Megra, Tidal Cycles, Maximilian, Punctual o Conduct.
\item Max/MSP
\item WebAudioAPI
\item ¿DAWs? 
\end{itemize}

\section{Legado Video}

\subsection{Fluxus y openGL}

Para el caso de la imagen, retomo la influencia que tiene en la comunidad de live coding y en mi experienciación del performance audiovisual con la computadora el papel que tuvo el desarrollo Fluxus\footnote{\url{https://gitlab.com/nebogeo/fluxus/}} de Dave Griffiths que se remonta al 2007. Una característica peculiar de este desarrollo es el uso de una sintaxis tipo LISP que recuerda a desarrollos musicales basados en este lenguaje de programación como OpenMusic. 

Detrás de Fluxus también cabe destacar la importancia de sistemas de renderización de gráficos por computadora como OpenGL, que actualmente, son el punto de partida de software de alto nivel involucrado con este proyecto como OpenFrameworks y la variante para el navegador, webGL, que implementa la librería Three.js 

% Pregunta, esto no tendría que ir en otros momentos? Tal vez disperso o tal vez en la introducción 


\begin{itemize}
\item OpenFrameworks 
\item TouchDesigner y plataformas como 
\item Processing
\item Hydra
\item Three.js 
\end{itemize}

\section{AV / Javascript}

Experiencia de Notas de Ausencia y de Panorama 

Estructuras compartidas fuera de un marco de trabajo como los que se han enunciado hasta el momento.

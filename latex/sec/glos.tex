
% Buscar mejores referencias que wikipedia 

\newglossaryentry{softwarelibre}
{
    name=Software Libre,
    description={es el software que respeta la libertad de los usuarios y la comunidad. A grandes rasgos, significa que los usuarios tienen la libertad de ejecutar, copiar, distribuir, estudiar, modificar y mejorar el software. Es decir, el ``software libre'' es una cuestión de libertad, no de precio. Para entender el concepto, piense en ``libre'' como en ``libre expresión'', no como en ``barra libre.''}}

\newglossaryentry{git}
                 {
                   name=Git,
    description={es un sistema distribuido, libre y de código abierto de control de versiones diseñado para manejar todo, desde proyectos muy grandes o pequeños con velocidad y eficiencia}}

\newglossaryentry{aprendizajeautomatico}
                 {
                   name=aprendizaje automático,
    description={Una forma de inteligencia artificial}}

\newglossaryentry{renderizacion}
                 {
                   name=Renderización,
                   text=renderización,
    description={Anglicismo de representación gráfica. Procedimiento para generar una imagen bidimensional o tridimensional por medio de la computadora}}


\newglossaryentry{javascript}
                 {
                   name=Javascript,
    description={es un lenguaje de programación ligero, interpretado, o compilado justo-a-tiempo (just-in-time) con funciones de primera clase.}}


\newglossaryentry{STEAM}
                 {
                   name=STEAM,
    description={Science, technology, engineering, arts and maths (Ciencia, tecnología, ingeniería, artes y matemáticas). }}

\newglossaryentry{ruffbox}
                 {
                   name=Ruffbox,
    description={es un sintetizador/sampleador sencillo con un sequenciador de pasos basado en texto que corrre en la web. Consultado en: \url{https://github.com/the-drunk-coder/ruffbox} el \today}}


\newglossaryentry{cmasmas}
                 {
                   name=C++,
    description={es un lenguaje de programación diseñado en 1979 por Bjarne Stroustrup. La intención de su creación fue extender al lenguaje de programación C mecanismos que permiten la manipulación de objetos. En ese sentido, desde el punto de vista de los lenguajes orientados a objetos, C++ es un lenguaje híbrido. Consultado en: \url{https://es.wikipedia.org/wiki/C++} el \today}}

\newglossaryentry{lisp}
                 {
                   name=Lisp,
    description={es una familia de lenguajes de programación de computadora de tipo multiparadigma con larga historia y una inconfundible y útil sintaxis homoicónica basada en la notación polaca. Consultado en: \url{https://es.wikipedia.org/wiki/Lisp} el \today}}



\newglossaryentry{tidal}
                 {
                   name=Tidal Cycles,
    description={(o 'Tidal') es un software libre/abierto escrito en Haskell. Tidal usa SuperCollider, otro software de código abierto para síntesis y I/O. Consultado en: \url{https://tidalcycles.org/} el \today}}


\newglossaryentry{webaudio}
                 {
                   name=Web Audio API,
    description={
provee un sistema poderoso y versatil para controlar audio en la Web, permitiendo a los desarrolladores escoger fuentes de audio, agregar efectos al audio, crear visualizaciones de audios, aplicar efectos espaciales (como paneo) y mucho más. Consultado en: \url{https://developer.mozilla.org/es/docs/Web/API/Web_Audio_API} el \today}}

\newglossaryentry{altonivel}
                 {
                   name=Lenguajes de programación de alto nivel,
                   text=lenguajes de programación de alto nivel, 
    description={Un lenguaje de programación de alto nivel se caracteriza por expresar los algoritmos de una manera adecuada a la capacidad cognitiva humana, en lugar de la capacidad con que los ejecutan las máquinas}}


\newglossaryentry{WASM}
                 {
                   name=WASM,
    description={WebAssembly (abreviado Wasm) está diseñado como un objetivo de compilación portátil para lenguajes de programación, lo que permite la implementación en la web para aplicaciones de cliente y servidor. Consultado en: \url{https://webassembly.org/} el \today.}}


\newglossaryentry{OSC}
                 {
                   name=OSC,
    description={OpenSoundControl (OSC) es una especificación de transporte de datos ( una codificación ) para la comunicación en tiempo real a través de mensajes entre aplicaciones y hardware. Consultado en: \url{https://ccrma.stanford.edu/groups/osc/index.html} el \today}}


\newglossaryentry{tonejs}
                 {
                   name=Tone.js,
    description={Tone.js es un marco de audio en la web para crear  música interactiva en el navegador. La arquitectura de Tone.js busca la familiaridad tanto para músicos como para programadores de audio para crear aplicaciones de audio basadas en la web. Consultado en: \url{https://tonejs.github.io/} el \today.}}


\newglossaryentry{threejs}
                 {
                   name=Three.js,
    description={El objetivo de este proyecto consiste en crear una librería 3D fácil de usar, ligera, multi-navegador, para múltiples propósitos. Consultado en: \url{https://github.com/mrdoob/three.js} el \today}}


\newglossaryentry{audioworklets}
                 {
                   name=AudioWorklets,
    description={La interfaz AudioWorklet de la Web Audio API es usada par sumplir scripts presonalizados de procesamiento de audio que se ejecutan en un hilo independiente para proporcionar procesamiento de audio de baja latencia. Consultado en: \url{https://developer.mozilla.org/en-US/docs/Web/API/AudioWorklet} el \today}}

\newglossaryentry{hydra}
                 {
                   name=Hydra,
    description={Hydra es un sintetizador de video y entorno de codificación en vivo (livecoding) que funciona sobre el navegador. Es de software libre y está pensado tanto para principiantes como expertos. Consultado en: \url{https://hydra.ojack.xyz/} el \today}}

\newglossaryentry{html}
                 {
                   name=HTML,
    description={(Lenguaje de Marcas de Hipertexto, del inglés HyperText Markup Language) es el componente más básico de la Web. Define el significado y la estructura del contenido web. Consultado en: \url{https://developer.mozilla.org/es/docs/Web/HTML} el \today}}

\newglossaryentry{openframeworks}
                 {
                   name=openFrameworks,
    description={ es una conjunto de herramientas de código abierto escritas en C++ diseñadas para asistir el proceso creativo, proveyendo un marco de trabajo simple e intuitivo par la experimentación. Consultado en: \url{https://openframeworks.cc/about/} el \today}}

\newglossaryentry{supercolliderjs}
                 {
                   name=SuperCollider.js,
    description={SuperCollider.js es una librería-cliente con todas las funciones y baterías incluídas para el servidor de síntesis de audio SuperCollider y el intérprete del lenguaje SuperCollider. Consultado en: \url{https://crucialfelix.github.io/supercolliderjs} el \today}}

\newglossaryentry{supercolliderweb}
                 {
                   name=SuperCollider.web,
    description={Aplicación web basada en Node.js para crear archivos de audio para SuperCollider, con integración para SoundCloud. Consultado en: \url{https://github.com/khilnani/supercollider.web} el \today}}

\newglossaryentry{estuary}
                 {
                   name=Estuary,
    description={Estuary es una plataforma para la colaboración y el aprendizaje a través de la codificación en vivo. Estuary permite experimentar con sonido, música e imágenes en un navegador web. Así mismo, Estuary reúne una colección curada de lenguajes de codificación en vivo en un solo entorno, sin el requisito de instalar software (que no sea un navegador web) y con soporte para agrupaciones artísticas en red (ya sea con gente participando en la misma sala o distribuida por todo el mundo). Consultado en \url{https://estuary.mcmaster.ca/} el \today}}

\newglossaryentry{flok}
                 { name=Flok, description={Editor P2P colaborativo basado en web para live codear música ygráficos. Los plugins REPL permiten ejecutar TidalCycles, SuperCollider, FoxDot y Mercury. Los plugins para web están orientados a lenguajes embebidos en el editor como Hydra y P5.js. Consultado en: \url{https://github.com/munshkr/flok} el \today}}

\newglossaryentry{navegador}
                 {
                   name=Navegador Web,
    description={Un navegador web (en español, web browser) es un software, aplicación o programa que permite el acceso a la Web, interpretando la información de distintos tipos de archivos y sitios web para que estos puedan ser vistos.}}


\newglossaryentry{ipc}
                 {
                   name=Interacción Persona-Computadora,
    description={La interacción persona-computadora o persona-ordenador (IPO) es la disciplina dedicada a diseñar, evaluar e implementar sistemas informáticos interactivos para el uso humano, y a estudiar los fenómenos relacionados más significativos.}}



\newglossaryentry{sonotexto}
                 {
                   name=SonoTexto,
    description={es una clase para grabar y reproducir buffers de audio. Esta clase se usa para improvisar con músicos que tocan instrumentos acústicos, así, es posible guardar pequeños buffers de los instrumentos al momento de la improvisación. Consultado en: \url{https://github.com/hvillase/sonotexto} el \today.}}


\newglossaryentry{jitlib}
                 {
                   name=JitLib,
    description={JitLib consiste en una serie de espacios reservados ( proxies del lado del servidor y del lado del cliente) y esquemas de acceso. Estos dos aspectos de espacio corresponden a inclusión y referencia, dependiendo de su contexto - aquií, los espacios reservados son como roles que tienen cierto comportamiento y que pueden ser llenados por ciertos objetos. Consultado en: \url{https://doc.sccode.org/Overviews/JITLib.html} el \today.}}


\newglossaryentry{rust}
                 {
                   name=Rust,
    description={Rust es un lenguaje de programación compilado, de propósito general y multiparadigma. Es un lenguaje de programación multiparadigmático que soporta programación funcional pura, por procedimientos, imperativa y orientada a objetos.}}


\newglossaryentry{camposonico}
                 {
                   name=Camposonico,
    description={Camposónico ( radio algorítmica ) es una aplicación para escuchar e intervenir paisajes sonoros y música experimental, haciendo posible la interacción con el sonido que va de una autorreproducción simple a la experimentación creativa a través del uso de código. Consultado en: \url{https://github.com/diegovdc/camposonico} el \today}}

\newglossaryentry{seis8s}
                 {
                   name=Seis8s,
    description={es un lenguaje de programación que permite la interacción en tiempo real con audio digital y conocimiento musical localizado, particularmente de músicas de Latinoamérica. Consultado en: \url{https://github.com/luisnavarrodelangel/seis8s} el \today}}


\newglossaryentry{livelab}
                 {
                   name=LiveLab,
    description={es una nueva herramienta que empodera a artistas y a presentadores de arte para encontrarse, crear, colaborar, ensayar y por último, producir performances en múltiples locaciones desde virtualmente cualquier parte del mundo. Este software inovador de colaboración por medio de video expenda el campo actual de ofertas al permitir a los usuarios personalizar los medios en caminos que mejor acomodan a sus necesidades. Consultado en: \url{https://github.com/ojack/LiveLab} el \today}}


\newglossaryentry{instrument}
                 {
                   name=INSTRUMENT,
    description={INSTRUMENT es una librería para livecodear música (beats, bajos, armonía, looping, efectos, ruteo de señal, síntesis, etc.) y para fungir como interfaz con instrumentos musicales y controladores dentro del entorno SuperCollider}}

%\newglossaryentry{}
%                 {
%                   name=,
%    description={}}

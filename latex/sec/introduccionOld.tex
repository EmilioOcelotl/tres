
%%----------------------------

% La propuesta es resumuir lo más posible esto
% Puntos de partida

%-----------------------------------------------

\chapter{Introducción}

%Descripción muy breve del proyecto 
%Motivación, algo breve.

% Motivación: Los límites de mi lenguaje ( de programación ) son los límites de mi mundo
% Distintas variantes y discusiones que devienen de esto 

La presente investigación es un bucle de investigación creación que estudia algunas prácticas performáticas, audiovisuales y experimentales con lenguajes de programación como instancias de conocimiento. De manera particular, señala un giro hacia las implicaciones de la interpretación audiovisual y la producción artística de obras para la web.

La estructura del documento es modular, cada subsección puede funcionar de manera independiente y puede ser leída como un artículo por sí mismo. La suma de estos módulos componen una investigación que puede leerse en lo general y en lo particular. 

A continuación, se esclarecen algunos puntos necesarios para la lectura de este documento. 

% La estructura y el resultado final del texto están pensados para la web,

\section{Contexto de escritura} % declaratoria de principios ? 

El contexto de escritura de este proyecto está marcado por una primera intención de desbordar laa investigación hacia la práctica de la música por computadora y la exploración con sonido. La generación de gráficos por computadora ha sido una extensión de esta exploración. Para el proyecto actual, representa un giro de acomplamiento que permite ampliar el rango práctico y reflexivo.

A manera de agente externo, la pandemia de COVID-19 ha marcado la vuelta definitiva de este proyecto hacia el navegador. Las experiencias particulares hacen patente las consecuencias que esto ha tenido para la investigación y la práctica artística. En relación a las preocupaciones centrales de esta investigación, el encierro y las posibilidades de la interconexión por medio de la web trajeron de vuelta la discusión sobre la presencia y la colaboración/interpretación/composición. 

\section{Obras}

El proyecto está vinculado por la realización tecnológica y práctica de piezas para el navegador que son objeto de la investigación. Las piezas son: anti, caso2 y caso3.

Estas piezas orbitan distintos lenguajes de programación y plataformas para la transformación de señales de audio y video y renderizacion de gráficos tridimensionales. El centro de esta orbitación es javascript.

% Para glosarios: \gls

\section{Software libre y de código abierto}

La perspectiva del softwarelibre como un motivo para la investigación y práctica artística. De igual manera al código creativo, es importante señalar las implicaciones cercanas pero en algunos momentos contradictorias como el código abierto. Finalmente, establecer una crítica a la perspectiva redentora del software libre que contextualice el uso de la computadora como una interlocutora socialmente constituída por el trabajo de personas involucradas con la programación y la reflexión imbricada con el software.

% Perspectiva práctica. Aquí puedo hablar de lo que escribe JD y tomar el caso de Roberto

De manera particular, entendemos que las implicaciones del software libre y de código abierto tienen repercusiones en los resultados sonoros y visuales y especialmente, en las formas de organización social, económica y social de las personas que se involucran con lo que García llama sistemas de producción musical y que define como:

\begin{quote}

  ``un sistema de producción musical cuyos productos surgen de la interacción de personas que colaboran de manera distribuida, con herramientas, modos de operación y circuitos de distribución que funcionan bajo un principio generalizado de compartición y circulación libre de la cultura.''\citep[p.~65]{jorgeDavid2021}

\end{quote}

Este proyecto parte de motivaciones que expresan formas de practicar e investigar con y sobre tecnología. Algunas de las metodologías de trabajo que parten del hacer pero que desde nuestra perspectiva, se pueden extender al pensamiento y a la reflexión, están presentes en formas de trabajo como DIY o DIWO. Del giro y la problematización de estas herramientas en términos del trabajo en contextos colaborativos situados en latinoamérica retomamos perspectivas de trabajo como la que plantea y problematiza un espacio como Platohedro:

\begin{quote}

  ``El buen vivir sumak kawsay, demanda, en esta globalidad de conocimiento, de un sumak yachay, un buen conocer, de los saberes (nuevos y viejos). Es por tanto necesario desarrollar el buen conocer, aquel que beneficia a todos, que crea un entorno rico y fértil para la vida cultural, social, económica, política. En definitiva, crear una matriz productiva basada en conocimiento común y abierto''\citep[p.~31]{platohedro}

\end{quote}

% Por acá también podría ir algo referente a la decolonialidad 

La extensión de la cultura libre a otros ámbitos como a investigación artística o a la producción musical es uno de los puntos de partida de esta investigación. 

\section{Lenguajes de programación e investigación} % Podría ir en sobre este documento ? 

Partimos de la problematización de la práctica de la programación desde los estudios del software y tomamos la perspectiva de Winnie Soon y Geoff Cox para hablar de la programación como práctica:

\begin{quote}

``Consideramos la programación como una práctica cultural dinámica y un fenómeno, una forma de estar y hacer en el mundo y un medio para entender algunos de los complejos procedimientos que sustentan y constituyen nuestras realidades vividas, y para actuar sobre esas realidades'' \citep[p.~14]{aestheticProgramming}
  
\end{quote}

% Qué papel tienen las "ciencias del espíritu" de cara a la investigación artística que se centra en un tivo de investigación de las "ciencias exactas / naturales" ? 

Es por esto que retomamos algunas ideas de las ciencias sociales y de la observación participante. Como desplazar el centro de la investigación al sonido, la imagen o su integración como objetos de conocimiento por sí mismos. En este sentido, la investigación busca adentrarse en el código como una tecnología que permite establecer un tipo de relación entre los agentes involucrados. El caso de \cite{diProspero} ilustra el interés por vincular una práctica técnica con procesos sociales: ``La sociabilización en la actividad del live coding, se constituye, como veremos, en consonancia con la técnica, por lo cual se hace necesario un análisis que dé cuenta de la relación de los sujetos de estudio con la tecnología.''\citep[p.~48]{diProspero} Si los actores humanos son los sujetos de estudio de una investigación que se centra en tecnología y partiendo del marco Latouriano ¿Será posible plantear a la tecnología como la instancia del pensamimento de otros sujetos de estudio? ¿Cuáles serían las diferencias de estas instancias con respecto a la tecnología que no necesariamente se adscribe a un paradigma creativo o de práctica artística?

La problematización del marco de investigación investigación como algo presente en el proceso de escritura, como un objetivo secundario que pone en contradicción al investigador inmerso en una actividad práctica. El rodeo como un camino necesario para encontrar rutas no convencionales.

La ejecución y existencia de estas piezas supone un ecosistema de pensamientos prácticas y de otras piezas cercanas en términos tecnológicos y artísticos. La observación de estas piezas en contexto nos permitirá considerarlas como instancias de conocimiento que permitan a su vez, observar al paradigma de los lenguajes de programación como tecnología. De esta manera podemos contemplar las agencias que estas instancias por sí mismas tienen y que se les imputan en términos de relaciones políticas, sociales y económicas.

Este documento dialoga con perspectivas y prácticas que involucran lenguajes de programación. Codificación creativa ( Creative Coding ) y el término educativo STEAM es una de ellas. En este sentido, la perspectiva educativa juega un papel importante como una extensión del paradigma que describimos. La presente investigación coincide con la aproximación artística y las observaciones críticas planteadas por Winnie Soon y Shelly Knotts con respecto a la programación:

\begin{quote}
  ``Esta aproximación estética no solamente incluye una introducción a la programación de manera práctica y creativa, sino también el cultivo de un espacio abierto donde sea posible discutir y reflexionar sobre la cultura computacional.''\citep[p.~87]{soonKnotts}

\end{quote}
  

Escalas de tiempo y la posibilidad de acercar/alejar la perspectiva y el contexto de lo que se investiga. La metáfora como una posibilidad para establecer trazos de conceptos entre hilos. 

% http://webtypography.net/

% Distribuited web of care y Taeyoon Choi 

\section{Versiones y repositorios}

El uso de git como una herramienta para la escritura y el control de versiones, no solamente de código sino también de la parte escrita de la investigación. La investigación abierta. Crítica a estas plataformas y qué posibilidades existen de preservación de este documento en el contexto de tecnologías como la \emph{cadena de bloques} (blockchain) y \emph{wayback machine}. 

Dos referencias que establecen las posibilidades del trabajo con tecnología y escritura a partir del trabajo colaborativo y distribuido: Artículo de PiranhaLab y el artículo de PullPush. 

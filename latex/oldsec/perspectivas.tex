%% Esto podría no ser necesariamente un capítulo sino una serie de apuntes distribuidos en la tesis. Esto para ser congruente con la idea del bucle investigación-creación en música con tecnología 

\chapter{Perspectivas de investigación}

\section{Giro de los nuevos medios}

\textit{Tres Estudios Abiertos} retoma esta incorporación, parte del giro de los nuevos medios y de los estudios del software \citep{manovichlanguage}.

\section{Estudios del software y código}

Como una extensión del punto de partida, la investigación se adscribe a la escritura con y sobre software \citep{aestheticProgramming}.

\section{Programación y práctica artística}

Atiende al papel que juega la experiencia subjetiva y las implicaciones políticas y sociales en la programación que se extiende y se posibilita por las prácticas artísticas \citep{speakingCode}. 

Investigaciones cercanas que resuelven problemas similares (hasta el momento):

% Aquí puede ir el apartado de el código como partitura y como ejecución y la relación con austin 


\section{El código como agente y como prótesis}

Carolina di Prospero y Bruno Latour como referencias para explicar esta relación. 

% Tesis doctoral de Roberto Cabezas. 

% La arquitectura propuesta en las siguientes líneas (como bien se ha mencionado anteriormente) no tiene el enfoque de la máquina como entidad creativa autónoma, sino más bien como herramienta de expansión cognitiva, en torno a las posibilidades del dominio epistémico y en función de las imágenes culturales osociales que intervienen en los procesos creativos del ser humano

% La computadora o la máquina como una prótesis y no como un agente autónomo. Es posible resolver esta distinción a partir de la idea del trabajo muerto ? 

% El uso del aprendizaje automático tiene una meta concreta, la cual es la expansión del dominio epistémico para la construcción de nuevos ejemplos en el conjunto de datos híbridos y la transformación del espacio conceptual.

% Esta discusión puede ir en el apartado de anti. La pregunta es si hay una sola meta en el aprendizaje automático 

% Los sistemas computacionales híbridos son entonces una propuesta recursiva computacional, en la que la simbiosis ocurre cuando consideramos al ser humano un organismo capaz de realizar computación, en donde el tiempo de computación puede ser interrumpido para valorar mediante la percepción la inclusión de nuevos datos al espacio conceptual. Al final los humanos son los que encuentran la semántica en la salida de las máquinas.

% Interesante lo del tiempo, ahí podemos agregar la cuestión de las magnitudes de tiempo y espacio

% En general el principal beneficio del desarrollo de técnicas y nuevas arquitecturas de aprendizaje profundo se ha visto en áreas de visión por computadora, salud o biología sintética, pero el arte y el diseño también se han beneficiado de estos avances, puesto que existen laboratorios de investigación totalmente dedicados a explorar —en específico— el aprendizaje profundo con fines creativos o, en su defecto, cómo utilizar la investigación artística para encontrar nuevas aplicaciones para estas técnicas de computación en la sociedad actual.

% En algún momento aparece pero no está citado aqui: el papel que tiene esta tecnología para la industria y en qué momentos industria y creación parecen lo mismo. Anti 

% El uso de las GAN’s para el diseño de sistemas computacionales, puede ser el detonante para el cambio de paradigma hacia los roles de las máquinas en procesos creativos. El proyecto anterior, aunque sus resultados sonverdaderamente impresionantes, mantiene el paradigma convencional de investigación: la máquina como entidad autónoma de creación, lo que de algunamanera no fomenta el avance del ser humano como un creador más completo o expandido.

% El papel de la previsualización

% La implementación de la GAN utilizada en CLUSTER se realizó con la librería Keras como interfaz simplificada de Tensorflow y, la arquitectura general fue tomada de (Mao, Li, Xie, Lau, Wang, & Smolley, 2017) con las llamadas Redes Adversarias Generativas de Mínimos Cuadrados (LSGAN’s).

% \printendnotes

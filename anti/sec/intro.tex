\iftoggle{tesis}
{La ofuscación puede definirse como el acto deliberado de encubrir el significado de una comunicación. Para el caso de la programación y apuntando ideas hacia los estudios del software, la presente investigación toma la noción de ofuscación de un conjunto de posibilidades para la escritura de software que coinciden, dialogan o se enfrentan a que podríamos definir como la convención de la \emph{estética del código} y aquellos programas que exploran ``otros principios estéticos'' además de los convencionales.

Edsger Dijkstra conoincide con la delimitación convencional de esta forma de escribir programas:

\begin{quote}
``[..] el programador no difiere de algún otro artesano: a menos de que ame sus herramientas, es altamente improbable que pueda crear algo de calidad superior. Al mismo tiempo estas consideraciones nos hablan de las más grandes virtudes que un programa puede mostrar: Elegancia y Belleza''\citep[p.~10]{EWD:EWD35}
\end{quote}

Como respuesta a la posición de Dijkstra y en un ámbito de programación que se aproxima lúdicamente a la escritura de programas, la ofuscación:

\begin{quote}

`` arroja luz a la naturaleza del código fuente, que es leído por un humano e interpretado por una máquina, y puede recordar a los críticos la búsqueda por diferentes dimensiones de sentido y múltiples codeos en todo tipo de programas''\citep[p.~198]{obfuscatedCode}

\end{quote}

El código su lectura, así como las funciones de los programas que ejecuta, son subjetivas y están determinadas por un sentido imputado que socialmente se acuerda de manera tácita y que puede ser visibilizado para interpelarlo en un sentido crítico, lúdico e incluso satírico.\footnote{Tal es el caso de Windows 93 del artista Jankenpopp. \url{https://www.windows93.net/} } En este punto encontramos conexiones con las posibilidades de la programación en una dimensión artística:

\begin{quote}
``[La práctica de la programación ofuscada] sugiere que el codeo puede resistir a la claridad y a la elegancia para pugnar en su lugar por la complejidad, puede hacer familiar lo desconocido y puede luchar con el lenguaje en el que está escrito, justo como lo hace la literatura contemporánea.''\citep[p.~198]{obfuscatedCode}
\end{quote}}
{La ofuscación puede definirse como el acto deliberado de encubrir el significado de una comunicación. Para el caso de la programación y apuntando ideas hacia los estudios del software, la presente investigación toma la noción de ofuscación de un conjunto de posibilidades para la escritura de software que coinciden, dialogan o se enfrentan a que podríamos definir como la convención de la \emph{estética del código} \citep{EWD:EWD35} y aquellos programas que exploran ``otros principios estéticos'' \citep{obfuscatedCode} además de los convencionales.}

¿Puede el código fuente ``luchar'' en contra del marco de uso para el que fue escrito?

% Para la versión expandida podría citar las referencias 

%% En este punto puedo retomar las ideas de djisktra y de knuth 

Esta definición es el punto de partida de \emph{Anti}, una pieza audiovisual para el navegador que tiene dos objetivos: visibilizar la discusión en torno a el uso de datos y la responsabilidad tecnológica del usuario y 2) actuar como un dispositivo de ofuscación facial y vocal que pueda utilizarse en situaciones de uso cotidiano. 

El maquillaje y el uso de accesorios anti-vigilancia son estrategias analógicas para evitar la detección de rostros. En una situación de protección fuera del entorno digital, incluso una máscara de leds puede cumplir esta función.

El presente proyecto se enfoca los mecanismos de anti-vigilancia que pueden realizarse de manera digital, teniendo a la computadora como un agente intermedio entre dos puntos que desean mantener algún tipo de comunicación gestual y vocal sin que estos puedan detectarse o asociarse a sujetos específicos, sin que esto implique que la comunicación sea completamente ofuscada para los usuarios.

% En la versión extendida para la tesis, aquí va el apartado de puesta en marcha y montaje. Considero que para el artículo puede ser innecesariamente extenso

